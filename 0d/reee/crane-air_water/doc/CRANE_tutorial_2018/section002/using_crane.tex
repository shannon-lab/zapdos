\subsection{Using CRANE}
	\begin{frame}[fragile]
		\frametitle{Action Description}
		\begin{columns}
			\column{0.45\textwidth}
			\begin{itemize}
				\item The focus of CRANE is the \textit{ChemicalReactions} Action, which accepts explicitly written reaction equations.
				\item The Action automatically adds all of the necessary Kernels, AuxKernels, UserObjects, etc. needed to solve equation (1).
				\item It includes two subblocks: [./Network] for spatial problems, and [./ScalarNetwork] for 0D problems.
			\end{itemize}
			\column{0.55\textwidth}
			\begin{Verbatim}[fontsize=\tiny]
[ChemicalReactions]
  [./ScalarNetwork]
    species = `e Ar+ Ar'
    file_location = 'RateCoefficients'

    reactions = `e + Ar -> e + e + Ar+      : EEDF
                 e + Ar+ + Ar -> Ar + Ar    : 1e-25'
  [../]
[]
			\end{Verbatim}
		\end{columns}
	\end{frame}

	\begin{frame}[fragile]
		\frametitle{Action Description}
		\begin{Verbatim}[fontsize=\tiny]
          [ChemicalReactions]
            [./ScalarNetwork]
              species = `e Ar+ Ar'
              file_location = `RateCoefficients'
              reactions = `e + Ar -> e + e + Ar+      : EEDF
                            e + Ar+ + Ar -> Ar + Ar    : 1e-25'
            [../]
          []
		\end{Verbatim}
		\vspace{-1em}
			\begin{itemize}
				\item {\footnotesize \textbf{species} - Denotes the active species in the reaction equations.}
				\begin{enumerate}
					\item {\scriptsize If a species appears in an equation but not in the ``species" parameter, it is ignored by the problem.}
					\item {\scriptsize All reactants must be either nonlinear or Auxiliary variables.}
				\end{enumerate}
				\item {\footnotesize \textbf{reactions} - Lists all reactions to be solved, with rate coefficients separated by a `:` character.}
				\item {\footnotesize \textbf{file\_location} - CRANE includes the option of having rate coefficients tabulated as a function of Auxiliary variables.}
				\begin{enumerate}
					\item {\scriptsize \textit{ChemicalReactions} looks up a corresponding file for each reaction with a `EEDF` rate coefficient. In this case, it will look for a file named `reaction\_e + Ar $->$ e + e + Ar+.txt` in the directory `RateCoefficients`.}
				\end{enumerate}
			\end{itemize}
	\end{frame}

	\begin{frame}[fragile]
		\frametitle{Action Description}
		\begin{Verbatim}[fontsize=\scriptsize]
[ChemicalReactions]
  [./ScalarNetwork]
    species = `e Ar+ Ar'
    file_location = `RateCoefficients'
    reactions = `e + Ar -> e + e + Ar+      : EEDF
                 e + Ar+ + Ar -> Ar + Ar    : 1e-25'
  [../]
[]
		\end{Verbatim}
			\begin{itemize}
				\item The Action automatically counts the stoichiometric coefficients and applies Kernels.
				\item This problem adds six total Kernels:
				\begin{enumerate}
					\item \textbf{e}: Product2BodyScalar (reaction 1), Reactant3BodyScalar (reaction 2)
					\item \textbf{Ar}: Reactant2BodyScalar (reaction 1), Product3BodyScalar (reaction 2)
					\item \textbf{Ar+}: Product2BodyScalar (reaction 1), Reactant3BodyScalar (reaction 2)
				\end{enumerate}
				\item Rate coefficients are included as Auxiliary variables, which are calculated at the beginning of each timestep
			\end{itemize}
	\end{frame}

	\begin{frame}[fragile]
		\frametitle{Rate Coefficients}
			\begin{itemize}
				\item Three rate coefficient formats are accepted:
				\begin{enumerate}
					\item \textbf{Constant} - denoted as a single value, e.g. $1e-25$
					\item \textbf{EEDF} - calculated externally by an EEDF software and tabulated. Denoted as `EEDF`.
					\item \textbf{Equation} - Explicit equations may be denoted by brackets. e.g. $\{(6.06e-6/Tgas)*exp(-15130.0/Tgas)\}$
				\end{enumerate}
				\item \textit{ChemicalReactions} automatically recognizes each format and adds the appropriate Auxiliary kernels.
				\item BOLSIG+ is the recommended Boltzmann solver for EEDF rate coefficients. An open source python equivalent, BOLOS, may also be used.
			\end{itemize}
	\end{frame}
	
	\begin{frame}[fragile]
	\frametitle{Tabulated Rate Coefficients}
	\begin{itemize}
		\item If an `EEDF' rate coefficient is included, CRANE will search for files of tabulated rate coefficients in the location specified by \textbf{file\_location}
		\item Rate coefficient files are typically stored in /problems/: \texttt{~/projects/crane/problems/[RateCoefficientsFile]}
		\item If no file is specified, CRANE will search the \texttt{~/projects/crane/problems} directory for the necessary files
		\item Files must be named after the reaction they are applied to, e.g.:
		\begin{itemize}
			\item {\scriptsize Reaction: \texttt{e + Ar -> e + e + Ar+   : EEDF}}
			\item {\scriptsize File: \texttt{reaction\_e + Ar -> e + e + Ar+.txt}}
		\end{itemize}
	\end{itemize}
	\end{frame}
	
	\begin{frame}[fragile]
	\frametitle{Tabulated Rate Coefficients (cont.)}
	\begin{itemize}
		\item Rate coefficient units are up to the user; however, make sure that the units are consistent with the units being used for the nonlinear variables in the system (if species densities are measured in $\# cm^{-3}$, rate coefficients must be $s^{-1}$, $cm^3 s^{-1}$, $cm^6 s^{-1}$.
		\item The sampling variable can be either electron temperature ($eV$) or reduced electric field ($V m^2$), both computed by BOLSIG+. 
		\begin{itemize}
			\item In either case, the sampling variable must be a nonlinear or auxiliary variable in the CRANE input file. 
		\end{itemize}
	\end{itemize}
	\end{frame}

	\begin{frame}[fragile]
		\frametitle{Kernels}
		\begin{itemize}
			\item The suite of Kernels available in CRANE includes treatment of one-, two-, and three-body reactions.
			\item $\nu$ is the stoichiometric coefficient, $k$ is the rate coefficient, and $n_i$ is the species concentration.
		\end{itemize}
		\resizebox{\linewidth}{!}
		{
		% \begin{table}
		% \centering
		% \caption{Scalar kernels included in CRANE. The sign of the stoichiometric coefficient, $\nu$, determines whether a product or reactant ScalarKernel is included.}
		\begin{tabular}{| c | c | c |}
		\hline
		\textbf{ScalarKernel Name} & \textbf{Governing Equation} & \textbf{Reaction Equation} \\
		\hline
		[Product/Reactant]1BodyScalar & $\nu k n_A$ & $A \xrightarrow{k} B$ \\
		\hline
		[Product/Reactant]2BodyScalar & $\nu k n_A n_B$ & $A + B \xrightarrow{k} C + D$\\
		\hline
		[Product/Reactant]3BodyScalar & $\nu k n_A n_B n_C$ & $A + B + C \xrightarrow{k} D + E + F$ \\
		\hline
	\end{tabular}
	}
	% \end{table}
	\end{frame}
	
	